%Metroplolis Beamer Theme: https://github.com/matze/mtheme
\documentclass[aspectratio=169, 10pt, dvipsnames, handout]{beamer}
\usetheme{metropolis}
\usepackage{appendixnumberbeamer, lmodern, bookmark,fontawesome}
\usepackage{booktabs}
% \usepackage[sorting=none]{biblatex}
\usepackage[scale=2]{ccicons}
\usepackage{pgfplots}
\usepgfplotslibrary{dateplot}
\usepackage{xspace}
\newcommand{\themename}{\textbf{\textsc{metropolis}}\xspace}
\usepackage{bbm}
\usepackage{tikz, graphicx}
\usepackage{caption, subcaption}
\usepackage{multicol}
% \usepackage[dvipsnames]{xcolor}
\usepackage{animate}
\usepackage{scalerel,xparse}
\usepackage[style=british]{csquotes}

\title{Kernels for proteins and MMD.}
% \subtitle{Lab Update}
\date{\today}
\author{Philip Hartout}

% \titlegraphic{\hfill\includegraphics[height=1.5cm]{logo.pdf}}

\hypersetup{
  colorlinks=true,
  linkcolor=orange,
  filecolor=orange,
  urlcolor=orange,
}

\def\signed #1{{\leavevmode\unskip\nobreak\hfil\penalty50\hskip1em
  \hbox{}\nobreak\hfill #1%
  \parfillskip=0pt \finalhyphendemerits=0 \endgraf}}

\newsavebox\mybox
\newenvironment{aquote}[1]
  {\savebox\mybox{#1}\begin{quote}\openautoquote\hspace*{-.7ex}}
  {\unskip\closeautoquote\vspace*{1mm}\signed{\usebox\mybox}\end{quote}}

\titlegraphic{%
  \includegraphics[width=.2\textwidth]{figures/mlcb-transparent.png}\hfill
  \includegraphics[width=.2\textwidth]{figures/dbsse-transparent.png}\hfill
  \includegraphics[width=.2\textwidth]{figures/eth-transparent.png}
}

\makeatletter
\setbeamertemplate{title page}{
  \begin{minipage}[b][\paperheight]{\textwidth}
    \vfill%
    \ifx\inserttitle\@empty\else\usebeamertemplate*{title}\fi
    \ifx\insertsubtitle\@empty\else\usebeamertemplate*{subtitle}\fi
    \usebeamertemplate*{title separator}
    \ifx\beamer@shortauthor\@empty\else\usebeamertemplate*{author}\fi
    \ifx\insertdate\@empty\else\usebeamertemplate*{date}\fi
    \ifx\insertinstitute\@empty\else\usebeamertemplate*{institute}\fi
    \vfill
    \ifx\inserttitlegraphic\@empty\else\inserttitlegraphic\fi
    \vspace*{1cm}
  \end{minipage}
}
\makeatother

\usetikzlibrary{shapes.geometric, arrows}

\tikzstyle{orangebox} = [rectangle, rounded corners, minimum width=2cm, minimum height=0.5cm, draw=black, fill=orange!40]
\tikzstyle{bluebox} = [rectangle, rounded corners, minimum width=2cm, minimum height=0.5cm, draw=black, fill=blue!40]

\tikzstyle{arrow} = [thick,->,>=stealth]


\begin{document}

\maketitle

\begin{frame}[fragile]{Introduction}
  \begin{itemize}
  \item Here we want to go through the maths behind the kernels to make sure our
    implementation is efficient.
  \item Linear, Weissfeiler-Lehmann kernel
  \end{itemize}
\end{frame}

\begin{frame}[fragile]{Linear kernel}
  \begin{itemize}
  \item Not much to optimize there. Just check:
    \href{https://scikit-learn.org/stable/modules/generated/sklearn.metrics.pairwise.linear_kernel.html}{this
    link}.
  \end{itemize}
\end{frame}


\begin{frame}[fragile]{W-L kernel}
  \begin{itemize}
  \item Acutally, W-L K. is a generalized version of the bag of node degrees
    since node degrees are one-hop neighborhood node enrichment while W-L
    iteratively enrich node voc. by doing $n$-hops.
  \item Steps: take input graph, (1) iteratively compute hash value of each node
    based on their neighborhood, (2) create feature map with counts of nodes
    with given color at each iteration of the hash value computation
  \item Notes: complexity is linear in \# edges. only the colors appearing in two
    graphs needs to be tracked. Counting colors can be done in linear time in \#
    nodes. In total, $\mathcal{O}(e)$
  \end{itemize}
\end{frame}


\begin{frame}[allowframebreaks]{References}

  \bibliography{../../thesis/refs.bib}
  \bibliographystyle{abbrv}

\end{frame}


\end{document}
