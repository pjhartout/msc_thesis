\chapter{Discussion}

% TODO Introduction

\section{Key Findings}

Overall, we found that MMD behaved surprisingly well on graphs extracted from
proteins (e.g. see Figure \ref{fig:mmd_consistent_eps}). This finding is
somewhat surprising considering recent findings indicating the relative
instability of MMD on synthetic graphs such as Watts-Strogatz graphs and
Barabási–Albert graphs under some combinations of kernels and descriptors
\citep{o2021evaluation}. While we certainly found unstable configurations
(see for instance Figure \ref{fig:mmd_consistent_eps} upper left pane with the degree histogram),
this was not the rule.

Furthermore, we found that the sensitivity of the MMD to perturbation was
greatly influenced by the graph representation extracted from the protein.
Namely, for $\varepsilon$ graphs (which overall seemed more stable than $k$-NN
graphs), the higher $\varepsilon$, the lower the sensitivity to low levels of
perturbation (Figure \ref{fig:mmd_sensitivity_eps}).

We introduced in Section \ref{sec:descriptors} and shown in Section
\ref{sec:results_protein_descriptors} two \emph{novel} protein-specific descriptors for
used in MMD, which are both extremely fast to compute (Table \ref{tab:runtimes})
and are able to detect perturbations with a surprising sensitivity (Figure
\ref{fig:protein_specific_descriptors}).

Furthermore, we found that the Weisfeiler-Lehman kernel, introduced in Section
\ref{sec:methods_kernels}, applied to MMD was able to detect perturbations
unable to be detected by traditional graph descriptors such as point mutations,
which are relevant for the protein domain (Section
\ref{sec:results_graph_kernels}). However, as with the rest of perturbations,
the Weisfeiler-Lehman kernel is not sensitive to lower regimes of perturbation
(see Figure \ref{fig:wlk}).
This can be alleviated by using an alternative descriptor such as the ESM
protein embedding introduced in Section \ref{sec:evalproblem}, which is more
sensitive to lower rates of point mutation (see Figure
\ref{fig:esm_descriptor}).

Finally, we analyzed the MMD obtained from kernels operating on persistence
diagrams \ref{sec:results_topo_kernels}, which also seemed very promising.
% TODO: complete

\section{Recommendations for the
practitioner}\label{sec:discussion_recommendations}

Using those key findings, we now formulate a set of recommendations for a
practitioner who needs to evaluate a generative protein model using one of the
protein representations highlighted above.

\subsection{The Right MMD For The Right Setting}\label{sec:discussion_right_mmd}
% Comment also on std dev - could potentially help with the p-value if high.

\subsection{Computational Requirements}
% Choose mmd config based on computational requirements

\subsection{Kernel Parameters}
% Kernel recommendations

\subsection{Practical Advice To Evaluate Generative Models}
% p-values, violin plots to visualize mmd.

\section{Limitations \& Future Directions}

\subsection{Negative Control}
% No'' `negative control'

\subsection{Expressive Power of Histograms}
% Discuss descriptors with bin_ranges not able to capture extreme cases

\subsection{Neural Network Pathologies On Unseen Data}
% Extrapolation is the issue
% ESM might work unpredictably on unseen sequences
% Mention unbiased kernels such as the spectrum kernel

\subsection{TDA Limitations}
% TDA on different point clouds
% computational complexity

\subsection{Mode Collapse \& Mode Dropping}

\section{Summary}
