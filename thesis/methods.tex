\chapter{Methods}

% Dummy text.% Don't forget about precomputign difference between x and y in
% % Gaussian kernels

The primary methodology employed in this thesis to assess the quality of metrics
used to evaluate generative protein models follows a fundamentally similar
approach used by \cite{o2021evaluation}, which we adapt to include the following
steps:

\begin{enumerate}
\item Take two i.i.d. samples a database of proteins.
\item Progressively add some perturbation to one of the samples.
\item Measure the MMD between the unperturbed and perturbed sample.
\end{enumerate}

We start by motivating the datasets that will be used to simulate a generative
protein model. We then describe and motivate the experimental setups -- i.e.
perturbations -- that we will employ to test the various configurations MMD.
Finally, we enumerate and justify which configurations of MMD are tested,
including all the combinations of protein representations, descriptor functions,
and kernels. Finally, we describe and motivate the experimental setups -- i.e.
perturbations -- that we will employ to test the various metric configurations.


\section{Datasets}

Because generative models have not been applied to proteins yet, which is partly
due to the lack of suitable evaluation metrics, no generative model can be used
here. In this thesis, except otherwise stated, all results will be dervied from
10 random samples from the \textit{homo sapiens} monomeric proteome downloaded
from the EBI AlphaFold2 database
\citep{varadi2022alphafold,tunyasuvunakool2021highly}, a repository comprising
predicted 3D structures of protein sequences obtained from AlphaFold2, the
current state-of-the-art method to predict protein structure from sequences
\citep{jumper2021highly}. This dataset was chosen because it contains
consistently formatted pdb files consistently only containing information from
heavy atom directly contributing to the 3D structure of a single monomer, which
simplifies downstream processing.

\section{Perturbations}

While \cite{o2021evaluation} focused on \emph{graph perturbations} specifically,
we wanted to augment and refine the set of perturbation applied to the perturbed
sample of proteins to be more pertinent to this particular data type. Three
categories of perturbations can be distinguished:
\begin{itemize}
\item \emph{Graph Perturbations}. These perturbations mostly overlap with those
defined by \cite{o2021evaluation}, as they include (i) adding edges to a graph
(ii) removing edges from a graph (iii) rewiring, i.e. swapping, edges within a
graph.
\item \emph{Point cloud perturbations}. These perturbations aim to add changes
to the underlying coordinates of each of the atom in the protein. Such
perturbations include (i) injecting Gaussian noise (ii) twisting (iii) shearing
(iv) tapering.
\item \emph{Mutation}. These simply consist in swapping node labels, i.e.
changing an existing node label, with another. While graph perturbation
probabilities (i.e. of adding an edge) ranges from 0 to 1, here we mostly
concentrate on lower regimes of mutation, i.e. between 0 and 0.1, so see how
sensitive the MMD to a few point mutations, which covers most relevant
real-world protein engineering use cases \cite{poluri2016protein}.
\end{itemize}

For each perturbation, a range of degree of perturbation is defined and 20
different degrees of perturbation is examined and repeated 10 times with i.i.d.
set of proteins to estimate the sensitivity of the particular MMD configuration
to the.

\section{MMD Configurations}

\section{Experimental Setup}

\section{Summary}
