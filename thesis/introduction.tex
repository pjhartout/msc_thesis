\chapter{Introduction}

% Target length: 1,5 pages.

Generative modelling is a highly active branch of machine learning aiming to
model the joint distribution of a certain feature set $x$ and target labels $y$,
often referred to as $p(x,y)$. Modelling this joint distribution presents a lot
of advantages in a myriad of domains, but can be especially
consequential in biology \citep{lopez2020enhancing,strokach2022deep}.

In protein science specifically, the application domains of generative models
can solve a myriad of relevant tasks, from novel protein sequence design, to
designing proteins with specific shapes or functions, etc
\citep{jendrusch2021alphadesign,madani2021deep}. Moreover, the resulting models generate
embeddings and/or weights which can be transferred to discriminative settings
where one seeks to predict $y$ given $x$, i.e. $p(y|x)$. In such a setting, even
more tasks can be tackled using generative models as a starting point, such as protein
structure prediction \citep{jumper2021highly}, function prediction
\citep{meier2021language}, stability prediction \citep{strokach2020fast}, etc.

However, the design and improvement of generative model designs applied to
proteins is prohibited by the lack of suitable evaluation metrics. It is
notoriously hard to find expressive, robust and efficient performance measures
that accurately gauge the quality of generated samples. Active efforts have been
made to solve this challenge in the image domain \citep{heusel2017gans}, and
recently, the community has investigated this challenge specifically for a
common representation of proteins: graphs \citep{thompson2022evaluation,
o2021evaluation}. In this domain, the standard measure used is a highly
versatile statistic for a kernel two-sample test called \acrfull{mmd}
\citep{gretton2012kernel}.

The results of these investigations, especially those conducted by
\cite{o2021evaluation} unveiled a number of pitfalls related to MMD. Depending
on the kernel and kernel parameter configuration, different MMD configurations
ranked the quality of samples generated by different models differently. In
addition, when progressively perturbing a set of synthetic graphs, certain MMD
configurations with respect to another set of unperturbed graphs issued from the
same distribution was not always found to monotonically increase with increasing
amounts of perturbations, casting doubt as to the expressivity of MMD in certain
configurations and data.

In this thesis, we set out to clarify the behaviour of various MMD
configurations on protein data sets. We investigate the behaviour of various MMD
configurations by applying perturbations to one set of proteins. These
perturbations are relevant for protein design use cases and include sequence
perturbations, graph perturbations and geometric perturbations. In this setting,
we investigate the behaviour of classical MMD configurations typically used to
evaluate generative models in the graph domain. In addition, we expand this set
of configurations by devising novel combinations of protein representations,
descriptors, and kernels, all relevant for different aspects of protein design,
and integrate them to the well-established MMD evaluation framework.

This thesis is organized as follows: first, chapter \ref{chap:background}
introduces fundamental concepts that we are going to build upon, and discusses
the surrounding literature. Chapter \ref{chap:methods} details the methodology
of the various experiments that we carry out in this thesis, as well as
describes all the configurations of MMD that we will explore (i.e. all
combinations of representations, descriptor functions, kernels, and kernel
parameters). Chapter \ref{chap:results} presents and contextualizes the findings of those
experiments. We summarize key findings, make recommendations for practitioners and
highlight some limitations and directions for future work in Chapter
\ref{chap:discussion} before concluding in Chapter \ref{chap:conclusion}.

% Graph representations encapsulates a wide variety of data. Show how they can
% be useful in biology.

% Leslie paper: even worse, depending on parametrization, different models come
% out on top. Call for principled metric design is open.

% Machine learning for graphs. Describe different tasks. Discriminative vs.
% generative modelling.

% Challenges with generative modelling assessment, especially with related to
% graphs vs images. Situate within proteins. Show that it builds on Obray et al.

% Thesis overview
