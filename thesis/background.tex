\chapter{Background \& Related Work}

This chapter introduces the core concepts built upon in this thesis and surveys
recent literature tackling the evaluation of generative graph neural networks
and the relevance of this problem in structural biology. Section
\ref{sec:background} defines core mathematical and biological concepts that will
be built upon in the thesis. Section \ref{sec:related} will discuss recent
advances in the design of measures used to evaluate generative graph neural
networks and in structural biology.

\section{Background}\label{sec:background}

The set of methods investigated in this thesis lies at the interface of
structural biology and machine learning. We start by defining some relevant
biological properties of proteins, followed by a survey various graph theoretical
abstractions derived from the protein structure. We then move on to define
generative models and the various classes of measures used to evaluate them.

\subsection{Proteins}

Proteins are large biomolecules that are formed from a sequence of amino acids,
performing their functions as determined by their three-dimensional structure, and amino
acid sequence. Proteins support a vast array of functions in living organisms,
such as catalysing metabolic reactions, DNA replication, providing structural
support to cells, transporting molecules and sensing stimuli.

Structurally, each protein is made up of one or more chains of amino acids,
each of which contain a backbone and different side chains. The atoms in the
backbone include a \textalpha-carbon,

\subsection{Graphs}

\subsection{Topological Data Analysis}

\subsection{Generative models}

\subsection{Kernel methods}

\section{Related Work}\label{sec:related}

\subsection{Structural Biology}

\subsection{Metrics for Generative Graph Models}

\section{Summary}
