\chapter{Conclusion}\label{chap:conclusion}

In this thesis, we performed a meta-evaluation of metrics based on \acrshort{mmd}. We first
examined known configurations of \acrshort{mmd}, specifically by extracting graphs and
using common descriptor functions such as the degree distribution, laplacian
spectrum and clustering coefficient histogram to obtain a fixed-length vector,
which we can then feed to appropriate kernels such as the linear kernel and the
Gaussian kernel.

We then expanded the configurations of \acrshort{mmd} to include a kernel that operates
directly on graphs: the Weisfeiler-Lehman kernel, which we optimize to decrease
runtime complexity on sparse graphs such as those used here. Furthermore, we
used topological kernels such as the multi-scale kernel and the persistence
Fisher kernel to capture topological features of the protein. Finally, we
designed two new protein-specific descriptors: the interatomic distance
histogram and the dihedral angle histogram, which were inspired by the
validation pipeline of Protein Data Bank.

Overall, we find that the majority of \acrshort{mmd} configurations behaved reasonably
well. We hypothesize that the highly structured nature of the graphs used here
explain why our \acrshort{mmd} configurations behaved more predictably than those
in \cite{o2021evaluation}'s perturbation experiments, where synthetic, more
unstructured graphs were investigated.

Furthermore, we found that the procedure by which graphs were extracted from
proteins had a sizeable influence on the resulting sensitivity of \acrshort{mmd} to lower
regimes of perturbations. For instance, a lower value of $\varepsilon$ results
in \acrshort{mmd} values that were more sensitive to perturbations.

While the Weisfeiler-Lehman kernel performed relatively poorly on the specific
graphs extracted in this thesis, we found that \acrshort{mmd} configurations leveraging
topological representations and the protein-specific descriptor functions
behaved well under all appropriate perturbation regimes. However, the runtime of
TDA-related workflows were substantially higher than the others, while the
protein-specific descriptors we introduced here were among the fastest to
compute.

The approach devised in this thesis has several limitations. First, a negative
control cannot always be established, i.e. a worst-case \acrshort{mmd} - practitioners then
have to be satisfied with a highly perturbed set of proteins as a proxy for poor
performance. One of the major drawbacks of \acrshort{mmd} is that it is in general not
sensitive to model pathologies that do not affect the average kernel similarity
between samples, i.e. when mode collapse occurs.

To conclude, \acrshort{mmd} forms the basis of a powerful computational platform to
evaluate generative models. Configured with a powerful combination of
representations, descriptor functions, and kernels, \acrshort{mmd} could hopefully
accelerate the design of even better generative models for protein design
applications and beyond.
