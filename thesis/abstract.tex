\begin{abstract}
Generative models applied to proteins are poised to revolutionize the \emph{in
silico} design of novel proteins satisfying various functional and topological
constraints. However, such models are notoriously hard to evaluate. The
\acrfull{mmd}, a statistic used in a kernel two-sample test, has emerged as a
highly versatile evaluation metric used to evaluate generative graph models.
This versatility is due to the fact that any kernel, and, therefore, any
appropriate underlying data representation, can be used. This makes it relevant
for proteins, because they can be represented as graphs, point clouds, and
sequences of amino acids. In this thesis, we aim to evaluate the applicability
of different representations, descriptor functions and kernel combinations for
use in \gls{mmd} to design relevant metrics to evaluate generative models
operating in the protein domain. Through a set of graph-based and point
cloud-based perturbation experiments, we first evaluate various configurations
of graph descriptors traditionally used to evaluate generative graph models.
Second, we expand the use of \gls{mmd} to encompass previously unused
configurations, such as (i) graph kernels, (ii) novel protein descriptor
functions tailored to evaluate structural properties of proteins such has the
dihedral angles histogram formed by each pair of amino acid, and the histogram
of pairwise distances between amino acids, and (iii) kernels operating on
topological descriptors of proteins. Using meta-metrics that accurately capture
the \emph{desiderata} of suitable metrics (expressivity, robustness and
efficiency), we find that there are multiple configurations of \gls{mmd} that
accurately gauge the quality of proteins.
\end{abstract}
