\begin{abstract}
Generative models applied proteins are poised to revolutionize the \emph{in
silico} design of novel proteins satisfying various functional and shape-related
constraints. However, such models are hard to evaluate -- determining how
realistic a generated set of proteins is represents one of the major challenges
in the field of generative models applied to proteins. For graphs, a
representation often used to investigate proteins, the community has gravitated
toward using a kernel two-sample test statistic called the maximum mean
discrepancy. While recent works highlighted some of its pitfalls, this statistic
remains highly versatile since any kernel can be used.

In this thesis, we start by evaluating the expressivity, robustness and
efficiency of maximum mean discrepancy on graph descriptors using graphs
directly extracted from proteins. We then expand the use of MMD by using novel
representations, descriptor functions, and kernels. Specifically, we investigate
the use of graph kernels in MMD. We also devise and apply novel protein-specific
descriptors, such as the dihedral angles histogram formed by each amino acid,
and the pairwise distance histogram of each amino acid, which are both highly
expressive and effecient. Finally, we apply a class of kernels able to operate
on persistence diagrams, a powerful topological descriptor leveraging the
shape-related information in the point clouds formed by each protein structure.


\end{abstract}
